\thispagestyle{empty}
\section*{Kurzdarstellung}
\label{sec:kurzdarstellung}
    
In dieser Ausarbeitung werden die Konzepte und Methoden der Objektserialisierung untersucht. Diese sind für die Effizienz und Zuverlässigkeit bei der Entwicklung der festgelegten Spieleideen von entscheidender Bedeutung. Beginnend mit einer Definition der Objektserialisierung, werten verschiedene Verfahren und Techniken dargelegt. Der Umgang mit referenzierten Daten und die Implementierung von Kompressionsmethoden zur Optimierung der Datenübertragung werden ebenfalls erklärt.

Ein weiterer Teil der Arbeit ist der Vergleich zwischen klassischen, strukturierten Datenformaten wie JSON und XML und effizienteren, auf Leistung ausgerichteten Formaten wie Protobuf, Cap’n Proto und Flatbuffers. Diese Diskussion dient als Grundlage für die Entscheidung, welche Bibliotheken für die beiden Projekte Park Chase und Into The Sky verwendet werden

Abschließend wird die Umsetzung der gewählten Verfahren für die beiden Projekte beschrieben. Neben einer ersten Implementierung sind hierbei auch die Testfälle wichtig, mit der die Funktionalität der Implementierungen sichergestellt werden.

